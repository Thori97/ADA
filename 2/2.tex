\documentclass[a4paper,11pt]{jsarticle}
\usepackage{amsmath, amsfonts}
\usepackage{bm}
\usepackage[dvipdfmx]{graphicx}
\begin{document}
  \title{先端データ解析論 第2回小レポート}
  \author{情報理工学系研究科電子情報学専攻M1 堀 紡希 48216444}
  \date{\today}
  \maketitle

  \section*{宿題1}
  ipynbファイルで別に出します.

  \section*{宿題2}
  $l_2$正則化回帰によってパラメータが以下のように表される.
  \begin{equation}
    \hat{\bm \theta}_i = ({\bm \Phi} ^\top _i
    {\bm \Phi }_i + \lambda {\bm I})^{-1} 
    {\bm \Phi }_i ^\top {\bm y}_i
  \end{equation}

  ただし,${\bm \Phi }_i$は$i$番目のデータを抜いた計画行列で,

  \[{\bm \Phi }^\top{\bm \Phi } = \sum _j {\bm \phi}_j{\bm \phi}_j^\top\]

  であるので,これは以下で表される.
  \[ {\bm \Phi} ^\top _i
  {\bm \Phi }_i = {\bm \Phi} ^\top 
  {\bm \Phi } - {\bm \phi}_i{\bm \phi}_i^\top  \]

  
  また,以下が成り立つ.
  \[{\bm \Phi }^\top {\bm y} -y_i \bm{\phi}_i = \sum _j {\bm \phi_j y_j} -y_i \bm{\phi}_i = {\bm \Phi}_i^\top y_i \]

  以上から,

  \begin{align*}
    \hat{\bm \theta}_i &= ({\bm U} - {\bm \phi}_i{\bm \phi}_i^\top)^{-1} ({\bm \Phi}^\top{\bm y} - y_i {\bm \phi}_i )\\
  \end{align*}

  そして逆行列の公式から$y_i$の予測値を以下のように計算できる.
  \begin{align*}
    {\bm \phi}_i^\top\hat{\bm \theta}_i &= {\bm \phi}_i^\top({\bm U} - {\bm \phi}_i{\bm \phi}_i^\top)^{-1} ({\bm \Phi}^\top{\bm y} - y_i {\bm \phi}_i )\\
    &= {\bm \phi}_i^\top \left( U^{-1} + \frac{U^{-1} {\bm \phi}_i {\bm \phi}_i ^\top U^{-1}}{1-{\bm \phi}_i ^\top U ^{-1} {\bm \phi}_i} \right)({\bm \Phi}^\top{\bm y} - y_i {\bm \phi}_i )\\
    &= \frac{{\bm \phi}_i ^\top U^{-1}({\bm \Phi}^\top {\bm y} - y_i {\bm \phi}_i)}{1-{\bm \phi}_i ^\top U ^{-1} {\bm \phi}_i}
  \end{align*}

  ここで, 
  \[ \tilde{\bm H} = \text{diag}(1-{\bm \phi}_0^\top U^{-1} {\bm \phi}_0, \, \dots, \, 1-{\bm \phi}_n^\top U^{-1} {\bm \phi}_n )\]

  \[{\bm H}{\bm y} = {\bm y} - {\bm \Phi} U^{-1}{\bm \Phi}^\top{\bm y} \]

  を用いると$\|\hat{\bm H}^{-1} {\bm H} {\bm y}\|^2/n$をノルムの計算によって以下のように計算できる.

  \[ \frac{1}{n} \|\hat{\bm H}^{-1} {\bm H} {\bm y}\|^2 = \frac{1}{n} \sum _{i=1} ^{n} \left( \frac{-y_i + {\bm \phi}_i^\top U^{-1}{\bm \Phi}{\bm y}}{1-{\bm \phi}_i ^\top U ^{-1} {\bm \phi}_i} \right) ^2\]

  一方で, 二乗誤差は, 
  \begin{align*}
    \frac{1}{n}\sum _{i=1}^{n} ({\bm \phi}_i^\top\hat{\bm \theta}_i - y_i) &= \frac{1}{n} \sum _{i=1} ^{n} \left( \frac{-y_i + {\bm \phi}_i^\top U^{-1}{\bm \Phi}{\bm y}}{1-{\bm \phi}_i ^\top U ^{-1} {\bm \phi}_i} \right) ^2
  \end{align*}

  であるので, 二乗誤差が, 
  \[ \frac{1}{n} \|\hat{\bm H}^{-1} {\bm H} {\bm y}\|^2\]
  であることが示された.
  
\end{document}