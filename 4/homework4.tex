\documentclass[a4paper,11pt]{jsarticle}
\usepackage{amsmath, amsfonts}
\usepackage{bm}
\usepackage[dvipdfmx]{graphicx}
\newcommand{\argmin}{\mathop{\rm arg~min}\limits}
\begin{document}
  \title{先端データ解析論 第4回小レポート}
  \author{情報理工学系研究科電子情報学専攻M1 堀 紡希 48216444}
  \date{\today}
  \maketitle

  \section*{宿題1}
  \begin{align*}
    J({\bm \theta}) &= \frac{1}{2} \sum _{i=1} ^n \widetilde{w}_i \left(f_{\bm \theta} ({\bm x}_i) -y_i\right)^2 \\
    &= \frac{1}{2} ({\bm \Phi } {\bm \theta} - {\bm y} )^{\top} \widetilde{{\bm W}} ({\bm \Phi } {\bm \theta} - {\bm y} )
  \end{align*}
  を最小化する${\bm \theta}$を求めればよい.
  ${\bm \theta}$で微分して, 
  \begin{align*}
    \frac{\partial J({\bm \theta})}{\partial {\bm \theta}} &= {\bm \Phi}^\top \widetilde{\bm W} {\bm \Phi} {\bm \theta} - \frac{1}{2}{\bm \Phi} \widetilde{\bm W}^\top {\bm y} - \frac{1}{2}{\bm \Phi} \widetilde{\bm W} {\bm y}\\
    &= {\bm \Phi}^\top \widetilde{\bm W} {\bm \Phi} {\bm \theta} - {\bm \Phi} \widetilde{\bm W} {\bm y}
  \end{align*}
  ただし,以下の式を用いた.
  \[ \frac{\partial}{\partial {\bm \theta}} {\bm \theta}^\top {\bm A} {\bm \theta} = 2{\bm A} {\bm \theta}, \frac{\partial}{\partial {\bm \theta}} {\bm b}^\top {\bm \theta} = {\bm b}, \widetilde{\bm W}^\top = \widetilde{\bm W} \]

  よって二乗誤差の最小を与えるパラメータ$\hat{\bm \theta}$はこれが$0$になるときで,以下で与えられる.

  \[ \hat{\bm \theta} = ( {\bm \Phi}^\top \widetilde{\bm W} {\bm \Phi} )^{-1} {\bm \Phi} \widetilde{\bm W} {\bm y}\]
  
  \section*{宿題2}

  二次上界$\widetilde{\rho} (r) $ は定数$a, b, c$を用いて以下のように表すことができる.

  \[ \widetilde{\rho} (r) = ar^2 + br + c \]

  $\rho (r)$は対称であるので以下が成り立つ.
  \[ \rho (\widetilde{r}) =  \rho (-\widetilde{r})\]

  $\widetilde{\rho} (r) $は$(\widetilde{r}, \rho (\widetilde{r})), (-\widetilde{r}, \rho (\widetilde{r}))$を通るので,代入して

  \[ b=0 \]

  を得る.また,上の点で接するので一次微分係数が一致する.すなわち

  \[ \rho ' (\widetilde{r}) = \widetilde{\rho} ' (\widetilde{r}) = 2a\widetilde{r} \]
  が成り立ち,$a$は以下の様に求まる.
  \[ a = \frac{\rho ' (\widetilde{r})}{2\widetilde{r}} \]

  以上より,存在するなら二次上界は
  \[ \widetilde{\rho} = \frac{\rho ' (\widetilde{r})}{2 \widetilde{r}}r^2 + \text{const}\]
  \section*{宿題3}
  別のipynbファイルで提出します.
\end{document}